\documentclass[french, a4paper, 12pt, twoside]{article}
\usepackage[utf8]{inputenc}
\usepackage[T1]{fontenc}
\usepackage[french]{babel}
\usepackage{graphicx}
\usepackage{float}
\usepackage{listings}
\usepackage{xcolor}
\usepackage{hyperref}
\usepackage{geometry}
\geometry{margin=2.5cm}

% Configuration pour le code Python
\lstset{
    language=Python,
    basicstyle=\ttfamily\small,
    keywordstyle=\color{blue},
    commentstyle=\color{gray},
    stringstyle=\color{red},
    numbers=left,
    numberstyle=\tiny\color{gray},
    stepnumber=1,
    frame=single,
    breaklines=true,
    captionpos=b
}

\title{\textbf{Mise en place d'une solution de gestion des déplacements} \\ Module Odoo 19}
\author{Ahmed Raji}
\date{15 Novembre 2025}

\begin{document}

\maketitle
\tableofcontents
\newpage

\section{Introduction}

\subsection{Contexte du projet}
La gestion des déplacements professionnels constitue un enjeu majeur pour les entreprises modernes. Elle implique la coordination de multiples acteurs (employés, managers, direction administrative et financière) et nécessite un suivi rigoureux des demandes, des validations et des coûts associés.

Ce projet vise à développer un module Odoo 19 dédié à la gestion complète du cycle de vie des demandes de déplacement, depuis leur création par les employés jusqu'à leur validation finale et leur clôture administrative.

\subsection{Objectifs du projet}
\begin{itemize}
    \item Automatiser le processus de demande de déplacement
    \item Implémenter un workflow de validation à trois niveaux (Employé → Manager → DAF)
    \item Calculer automatiquement les frais de mission selon des règles métier prédéfinies
    \item Assurer la traçabilité complète des actions et décisions
    \item Gérer les notifications et communications entre les acteurs
    \item Garantir la sécurité et les droits d'accès appropriés
\end{itemize}

\subsection{Technologies utilisées}
\begin{itemize}
    \item \textbf{Odoo 19.0} — Plateforme ERP open-source
    \item \textbf{Python 3} — Langage de programmation pour les modèles métier
    \item \textbf{XML} — Format pour les vues, données et templates
    \item \textbf{PostgreSQL} — Base de données relationnelle
    \item \textbf{Docker} — Conteneurisation pour l'environnement de développement
\end{itemize}

\section{Conception}

\subsection{Diagramme de classes}

\begin{figure}[H]
    \centering
    % \includegraphics[width=\textwidth]{images/classe.png}
    \caption{Diagramme de classes du module gestionDeplacements}
    \label{fig:diagramme_classes}
\end{figure}

\subsection{Diagramme de cas d'utilisation}

\begin{figure}[H]
    \centering
    % \includegraphics[width=0.9\textwidth]{images/use_case.png}
    \caption{Diagramme de cas d'utilisation du système de gestion des déplacements}
    \label{fig:diagramme_use_case}
\end{figure}

Le diagramme de cas d'utilisation illustre les interactions entre les trois types d'acteurs et le système. Les acteurs héritent de leurs droits de manière hiérarchique : le DAF possède tous les droits du Manager, qui lui-même possède tous les droits de l'Employé.

\subsubsection{Acteurs}
\begin{itemize}
    \item \textbf{Employé} — Utilisateur de base qui crée et soumet des demandes
    \item \textbf{Manager} — Validateur de premier niveau (hérite des droits Employé)
    \item \textbf{DAF} — Direction Administrative et Financière, validateur final (hérite des droits Manager)
\end{itemize}

\subsubsection{Cas d'utilisation par acteur}

	extbf{Employé :}
\begin{itemize}
    \item \texttt{Créer une demande} — Initialiser une nouvelle demande de déplacement
    \item \texttt{Soumettre une demande} — Envoyer la demande pour validation
    \item \texttt{Consulter ses demandes} — Visualiser l'état de ses demandes
    \item \texttt{Annuler une demande} — Retirer une demande soumise
\end{itemize}

	extbf{Manager (+ droits Employé) :}
\begin{itemize}
    \item \texttt{Valider une demande} — Approuver une demande soumise par un membre de l'équipe
    \item \texttt{Refuser une demande} — Rejeter une demande avec motif obligatoire
\end{itemize}

	extbf{DAF (+ droits Manager + Employé) :}
\begin{itemize}
    \item \texttt{Traiter une demande} — Prendre en charge une demande approuvée
    \item \texttt{Terminer une demande} — Clôturer définitivement le processus
    \item \texttt{Gérer la configuration} — Administrer les villes et véhicules de service
\end{itemize}

\subsection{Description des classes}

\subsubsection{Classe \texttt{hr.employee} (Odoo existante)}
\textbf{Rôle :} Représente les employés de l'entreprise. Cette classe fait partie du module HR d'Odoo.\\
\textbf{Attributs principaux :}
\begin{itemize}
    \item \texttt{name : String} — nom complet de l'employé
    \item \texttt{user\_id : Many2one} — lien vers le compte utilisateur
    \item \texttt{parent\_id : Many2one} — référence au manager direct
    \item \texttt{work\_email : String} — email professionnel
\end{itemize}
\textbf{Relations :}
\begin{itemize}
    \item Association 1..* avec \texttt{DemandeDeplacement} — un employé peut créer plusieurs demandes
    \item Auto-référence via \texttt{parent\_id} pour la hiérarchie managériale
\end{itemize}

\subsubsection{Classe \texttt{gestion.deplacement.demande}}
\textbf{Rôle :} Classe centrale du système. Elle représente une demande de déplacement avec toute sa logique métier.\\
\textbf{Attributs principaux :}
\begin{itemize}
    \item \texttt{name : String} — référence unique (ex: DEP00001)
    \item \texttt{employee\_id : Many2one} — employé demandeur
    \item \texttt{manager\_id : Many2one} — manager validateur (calculé automatiquement)
    \item \texttt{date\_debut : Date} — date de début du déplacement
    \item \texttt{date\_fin : Date} — date de fin du déplacement
    \item \texttt{nb\_jours : Integer} — nombre de jours (calculé)
    \item \texttt{destination\_city\_id : Many2one} — ville de destination
    \item \texttt{is\_international : Boolean} — indicateur de déplacement international (calculé)
    \item \texttt{mode\_transport : Selection} — moyen de transport choisi
    \item \texttt{distance\_estimee : Float} — distance en kilomètres
    \item \texttt{classe\_voyage : Selection} — classe de voyage pour l'avion (calculé)
    \item \texttt{vehicule\_id : Many2one} — véhicule de service si applicable
    \item \texttt{montant\_frais : Monetary} — montant des frais estimés (calculé)
    \item \texttt{mission\_objet : Text} — description de la mission
    \item \texttt{ordre\_mission\_file : Binary} — fichier PDF de l'ordre de mission
    \item \texttt{state : Selection} — état du workflow
    \item \texttt{motif\_refus : Text} — raison du refus si applicable
\end{itemize}

\textbf{États du workflow :}
\begin{itemize}
    \item \texttt{brouillon} — Demande en création par l'employé
    \item \texttt{soumis} — Demande soumise au manager
    \item \texttt{approuve} — Demande approuvée par le manager
    \item \texttt{en\_cours} — Demande prise en charge par la DAF
    \item \texttt{termine} — Demande terminée et archivée
    \item \texttt{refuse} — Demande refusée par le manager
    \item \texttt{annule} — Demande annulée
\end{itemize}

\textbf{Héritage Odoo :}
\begin{itemize}
    \item \texttt{mail.thread} — Active le Chatter pour les messages et historique
    \item \texttt{mail.activity.mixin} — Active le système d'activités et notifications
\end{itemize}

\textbf{Relations :}
\begin{itemize}
    \item Many2one avec \texttt{hr.employee} (employé demandeur)
    \item Many2one avec \texttt{hr.employee} (manager validateur)
    \item Many2one avec \texttt{gestion.deplacement.ville} (destination)
    \item Many2one avec \texttt{gestion.deplacement.service.vehicule} (optionnel)
\end{itemize}

\subsubsection{Classe \texttt{gestion.deplacement.ville}}
\textbf{Rôle :} Représente les villes de destination possibles.\\
\textbf{Attributs :}
\begin{itemize}
    \item \texttt{name : String} — nom de la ville
    \item \texttt{country\_id : Many2one} — pays de la ville
    \item \texttt{code\_postal : String} — code postal
    \item \texttt{active : Boolean} — indicateur actif/archivé
\end{itemize}
\textbf{Contraintes :}
\begin{itemize}
    \item Contrainte unique sur (name, country\_id) — évite les doublons
\end{itemize}
\textbf{Relations :}
\begin{itemize}
    \item Many2one avec \texttt{res.country} (classe Odoo existante)
    \item One2many inverse avec \texttt{DemandeDeplacement}
\end{itemize}

\subsubsection{Classe \texttt{gestion.deplacement.service.vehicule}}
\textbf{Rôle :} Représente les véhicules de service de l'entreprise.\\
\textbf{Attributs :}
\begin{itemize}
    \item \texttt{name : String} — nom du véhicule
    \item \texttt{immatriculation : String} — numéro d'immatriculation
    \item \texttt{marque : String} — marque du véhicule
    \item \texttt{modele : String} — modèle du véhicule
    \item \texttt{company\_id : Many2one} — société propriétaire
    \item \texttt{active : Boolean} — indicateur actif/archivé
\end{itemize}
\textbf{Relations :}
\begin{itemize}
    \item One2many inverse avec \texttt{DemandeDeplacement}
\end{itemize}

\subsubsection{Classe \texttt{gestion.deplacement.demande.refus.wizard}}
\textbf{Rôle :} Modèle transitoire (wizard) pour la saisie du motif de refus.\\
\textbf{Type :} TransientModel — données temporaires non persistées\\
\textbf{Attributs :}
\begin{itemize}
    \item \texttt{motif\_refus : Text} — raison du refus (obligatoire)
    \item \texttt{demande\_id : Many2one} — demande concernée
\end{itemize}
\textbf{Méthode :}
\begin{itemize}
    \item \texttt{action\_confirm\_refus()} — Applique le refus avec le motif saisi
\end{itemize}

\subsubsection{Énumération \texttt{mode\_transport}}
\textbf{Rôle :} Définit les modes de transport possibles.\\
\textbf{Valeurs :}
\begin{itemize}
    \item \texttt{train} — Train
    \item \texttt{autocar} — Autocar
    \item \texttt{avion} — Avion
    \item \texttt{vehicule\_service} — Véhicule de service
\end{itemize}

\subsubsection{Énumération \texttt{classe\_voyage}}
\textbf{Rôle :} Définit les classes de voyage pour l'avion (calculé automatiquement).\\
\textbf{Valeurs :}
\begin{itemize}
    \item \texttt{economique} — Classe économique (distance < 6000 km)
    \item \texttt{business} — Classe business (distance $\geq$ 6000 km)
\end{itemize}

\subsection{Relations clés résumées}

\begin{table}[H]
\centering
\renewcommand{\arraystretch}{1.3}
\begin{tabular}{|p{6cm}|p{2cm}|p{7cm}|}
\hline
\textbf{Relation} & \textbf{Type} & \textbf{Description} \\ \hline
Employee → DemandeDeplacement & 1..* & Un employé peut créer plusieurs demandes \\ \hline
Manager → DemandeDeplacement & 1..* & Un manager valide plusieurs demandes \\ \hline
DemandeDeplacement → Ville & 1..1 & Chaque demande a une destination \\ \hline
Ville → Pays & 1..1 & Chaque ville appartient à un pays \\ \hline
DemandeDeplacement → Vehicule & 0..1 & Optionnel si mode = véhicule de service \\ \hline
DemandeDeplacement → ModeTransport & 1..1 & Mode de transport obligatoire \\ \hline
\end{tabular}
\caption{Principales relations entre les classes}
\label{tab:relations}
\end{table}

\subsection{Distinction entre classes Odoo et personnalisées}

\begin{table}[H]
\centering
\renewcommand{\arraystretch}{1.3}
\begin{tabular}{|p{5cm}|p{10cm}|}
\hline
\textbf{Type} & \textbf{Description} \\ \hline
\textbf{hr.employee} & Classe Odoo existante du module HR \\ \hline
\textbf{res.country} & Classe Odoo existante pour les pays \\ \hline
\textbf{res.currency} & Classe Odoo existante pour les devises \\ \hline
\textbf{res.company} & Classe Odoo existante pour les sociétés \\ \hline
\textbf{mail.thread} & Mixin Odoo pour le Chatter \\ \hline
\textbf{mail.activity.mixin} & Mixin Odoo pour les activités \\ \hline
\hline
\textbf{gestion.deplacement.demande} & Classe personnalisée créée \\ \hline
\textbf{gestion.deplacement.ville} & Classe personnalisée créée \\ \hline
\textbf{gestion.deplacement.service.vehicule} & Classe personnalisée créée \\ \hline
\textbf{gestion.deplacement.demande.refus.wizard} & Classe personnalisée créée \\ \hline
\end{tabular}
\caption{Distinction entre classes Odoo existantes et personnalisées}
\label{tab:classes_odoo}
\end{table}

\subsection{Règles métier automatisées}

\subsubsection{Calcul du nombre de jours}
Le nombre de jours est calculé automatiquement à partir des dates de début et de fin :
\begin{lstlisting}[caption=Calcul automatique du nombre de jours]
@api.depends('date_debut', 'date_fin')
def _compute_nb_jours(self):
    for record in self:
        if record.date_debut and record.date_fin:
            delta = record.date_fin - record.date_debut
            record.nb_jours = delta.days + 1 if delta.days >= 0 else 1
\end{lstlisting}

\subsubsection{Détermination du caractère international}
Le système détermine automatiquement si le déplacement est international en comparant le pays de destination avec le pays de la société :
\begin{lstlisting}[caption=Détection automatique des déplacements internationaux]
@api.depends('destination_city_id', 'destination_city_id.country_id')
def _compute_is_international(self):
    company_country = self.env.company.country_id
    for record in self:
        if record.destination_city_id and record.destination_city_id.country_id and company_country:
            record.is_international = (record.destination_city_id.country_id.id != company_country.id)
\end{lstlisting}

\subsubsection{Calcul des frais de mission}
Le montant des frais est calculé selon des barèmes différents pour les déplacements nationaux et internationaux :
\begin{itemize}
    \item \textbf{National :} 700 MAD/jour
    \item \textbf{International :} 1500 MAD/jour
\end{itemize}

\begin{lstlisting}[caption=Calcul automatique des frais]
@api.depends('nb_jours', 'is_international')
def _compute_montant_frais(self):
    for record in self:
        if record.is_international:
            record.montant_frais = record.nb_jours * 1500.0
        else:
            record.montant_frais = record.nb_jours * 700.0
\end{lstlisting}

\subsubsection{Calcul de la classe de voyage}
Pour les déplacements en avion, la classe est déterminée automatiquement selon la distance :
\begin{itemize}
    \item \textbf{Distance < 6000 km :} Classe économique
    \item \textbf{Distance $\geq$ 6000 km :} Classe business
\end{itemize}

\begin{lstlisting}[caption=Calcul automatique de la classe avion]
@api.depends('distance_estimee', 'mode_transport')
def _compute_classe_voyage(self):
    for record in self:
        if record.mode_transport == 'avion':
            if record.distance_estimee > 6000:
                record.classe_voyage = 'business'
            else:
                record.classe_voyage = 'economique'
\end{lstlisting}

\subsection{Contraintes de validation}

Le système implémente plusieurs contraintes pour garantir la cohérence des données :

\begin{table}[H]
\centering
\renewcommand{\arraystretch}{1.3}
\begin{tabular}{|p{5cm}|p{10cm}|}
\hline
\textbf{Contrainte} & \textbf{Description} \\ \hline
Cohérence des dates & La date de fin doit être $\geq$ date de début \\ \hline
Distance positive & La distance doit être > 0 km \\ \hline
Règle avion & L'avion nécessite une distance $\geq$ 500 km \\ \hline
Véhicule obligatoire & Si mode = véhicule\_service, un véhicule doit être sélectionné \\ \hline
Ordre de mission & Obligatoire avant soumission \\ \hline
Employé unique & Un employé ne peut créer des demandes que pour lui-même \\ \hline
\end{tabular}
\caption{Contraintes de validation métier}
\label{tab:contraintes}
\end{table}

\section{Sécurité et Droits d'Accès}

\subsection{Groupes de sécurité}

Le module définit trois niveaux de sécurité hiérarchiques :

\subsubsection{Groupe 1 : Employé (group\_deplacement\_employee)}
\textbf{Droits :}
\begin{itemize}
    \item Créer ses propres demandes de déplacement
    \item Modifier ses demandes en état \texttt{brouillon}
    \item Soumettre ses demandes pour validation
    \item Consulter uniquement ses propres demandes
    \item Annuler ses demandes
    \item Remettre en brouillon les demandes refusées
\end{itemize}

\subsubsection{Groupe 2 : Manager (group\_deplacement\_manager)}
\textbf{Droits (en plus de ceux de l'employé) :}
\begin{itemize}
    \item Consulter les demandes de son équipe
    \item Approuver ou refuser les demandes soumises
    \item Accéder au wizard de refus pour saisir un motif
\end{itemize}

\subsubsection{Groupe 3 : DAF (group\_deplacement\_daf)}
\textbf{Droits (en plus de ceux du manager) :}
\begin{itemize}
    \item Consulter toutes les demandes de l'entreprise
    \item Prendre en charge les demandes approuvées
    \item Marquer les demandes comme terminées
    \item Gérer la configuration (villes, véhicules)
    \item Accès complet en lecture/écriture/suppression
\end{itemize}

\subsection{Record Rules}

\begin{table}[H]
\centering
\renewcommand{\arraystretch}{1.3}
\begin{tabular}{|p{4cm}|p{7cm}|p{4cm}|}
\hline
\textbf{Groupe} & \textbf{Domaine} & \textbf{Accès} \\ \hline
Employé & \texttt{[('employee\_id.user\_id', '=', user.id)]} & Ses demandes uniquement \\ \hline
Manager & \texttt{[('manager\_id.user\_id', '=', user.id)]} & Demandes de son équipe \\ \hline
DAF & \texttt{[(1, '=', 1)]} & Toutes les demandes \\ \hline
\end{tabular}
\caption{Règles de sécurité par groupe}
\label{tab:record_rules}
\end{table}

\section{Workflow et Notifications}

\subsection{Diagramme de flux du workflow}

Le processus de validation suit un workflow séquentiel à trois niveaux :

\begin{verbatim}
[Brouillon] --Soumettre--> [Soumis] --Approuver--> [Approuvé]
                               |
                           Refuser
                               |
                               v
                           [Refusé] --Remettre en brouillon--> [Brouillon]

[Approuvé] --Prendre en charge--> [En cours] --Terminer--> [Terminé]
\end{verbatim}

\subsection{Actions et transitions}

\begin{table}[H]
\centering
\renewcommand{\arraystretch}{1.3}
\begin{tabular}{|p{3cm}|p{3cm}|p{3cm}|p{5cm}|}
\hline
\textbf{Action} & \textbf{État initial} & \textbf{État final} & \textbf{Acteur} \\ \hline
Soumettre & brouillon & soumis & Employé \\ \hline
Approuver & soumis & approuve & Manager \\ \hline
Refuser & soumis/approuve & refuse & Manager \\ \hline
Prendre en charge & approuve & en\_cours & DAF \\ \hline
Terminer & en\_cours & termine & DAF \\ \hline
Annuler & * & annule & Employé/DAF \\ \hline
Remettre en brouillon & refuse & brouillon & Employé \\ \hline
\end{tabular}
\caption{Actions du workflow}
\label{tab:workflow_actions}
\end{table}

\subsection{Système de notifications}

Le module utilise trois mécanismes de notification complémentaires :

\subsubsection{Activités (horloge)}
\textbf{Rôle :} Notifications de tâches pour les actions requises\\
\textbf{Utilisation :}
\begin{itemize}
    \item Notification au manager lors de la soumission d'une demande
    \item Notification à tous les DAF lors de l'approbation d'une demande
    \item Affichage dans l'icône (horloge) avec compteur rouge
\end{itemize}

\begin{lstlisting}[caption=Création d'une activité pour le manager]
self.activity_schedule(
    'mail.mail_activity_data_todo',
    user_id=self.manager_id.user_id.id,
    summary=f"Demande de deplacement a valider: {self.name}",
    note=f"L'employe {self.employee_id.name} a soumis une demande."
)
\end{lstlisting}

\subsubsection{Chatter (Messages)}
\textbf{Rôle :} Historique complet des actions et communications\\
\textbf{Utilisation :}
\begin{itemize}
    \item Enregistrement de chaque transition d'état
    \item Messages système avec émojis pour meilleure lisibilité
    \item Visible par tous les followers de la demande
\end{itemize}

\begin{lstlisting}[caption=Post de message dans le Chatter]
self.message_post(
    body="Demande soumise pour validation par le manager",
    subtype_xmlid='mail.mt_comment'
)
\end{lstlisting}

\subsubsection{Emails}
\textbf{Rôle :} Notifications externes par email (optionnel)\\
\textbf{Templates disponibles :}
\begin{itemize}
    \item \texttt{mail\_template\_demande\_submitted} — Notification de soumission au manager
    \item \texttt{mail\_template\_demande\_approved} — Notification d'approbation à l'employé
    \item \texttt{mail\_template\_demande\_approved\_daf} — Notification de traitement aux DAF
    \item \texttt{mail\_template\_demande\_refused} — Notification de refus à l'employé
\end{itemize}

\textbf{Gestion des erreurs :}
\begin{lstlisting}[caption=Envoi sécurisé d'emails avec gestion d'erreurs]
try:
    template.send_mail(self.id, force_send=True)
except Exception as e:
    self.message_post(body=f"Erreur lors de l'envoi de l'email: {str(e)}")
\end{lstlisting}

\subsection{Système de followers}

Le système ajoute automatiquement les parties prenantes comme followers :
\begin{itemize}
    \item Le \textbf{manager} est ajouté lors de la soumission
    \item Tous les \textbf{DAF} sont ajoutés lors de l'approbation
    \item Les followers reçoivent les mises à jour du Chatter
\end{itemize}

\begin{lstlisting}[caption=Ajout automatique de followers]
self.message_subscribe(partner_ids=[self.manager_id.user_id.partner_id.id])
\end{lstlisting}

\section{Réalisation}

\subsection{Structure du module}

\begin{verbatim}
gestionDeplacements/
|-- __init__.py
|-- __manifest__.py
|-- models/
|   |-- __init__.py
|   |-- demande_deplacement.py
|   |-- ville.py
|   `-- service_vehicule.py
|-- views/
|   |-- demande_deplacement_view.xml
|   |-- ville_view.xml
|   |-- service_vehicule_view.xml
|   `-- menu.xml
|-- wizard/
|   |-- __init__.py
|   |-- demande_refus_wizard.py
|   `-- demande_refus_wizard_view.xml
|-- security/
|   |-- deplacement_security.xml
|   `-- ir.model.access.csv
|-- data/
|   |-- sequence.xml
|   |-- ville_data.xml
|   |-- service_vehicule_data.xml
|   `-- email_templates.xml
`-- static/
    `-- description/
\end{verbatim}

\subsection{Interface utilisateur : État Brouillon}

\begin{figure}[H]
    \centering
    % \includegraphics[width=0.95\textwidth]{images/brouillon.png}
    \caption{Interface de création d'une nouvelle demande (état Brouillon)}
    \label{fig:brouillon}
\end{figure}

L'interface de création permet à l'employé de saisir toutes les informations nécessaires :
\begin{itemize}
    \item \textbf{Informations générales :} Employé (auto-rempli), Manager (calculé), Dates
    \item \textbf{Destination :} Ville de destination (avec recherche), Distance
    \item \textbf{Transport :} Mode de transport, Véhicule (si applicable)
    \item \textbf{Ordre de mission :} Upload du fichier PDF obligatoire
    \item \textbf{Mission :} Description de l'objet de la mission
\end{itemize}

Les champs calculés (Nombre de jours, Montant estimé, Classe voyage) se mettent à jour automatiquement.

\subsection{État Soumis}

\begin{figure}[H]
    \centering
    % \includegraphics[width=0.95\textwidth]{images/soumis.png}
    \caption{Demande soumise pour validation}
    \label{fig:soumis}
\end{figure}

Après soumission :
\begin{itemize}
    \item L'état passe à \textbf{Soumis}
    \item Le formulaire devient en lecture seule
    \item Le manager reçoit une activité de notification
    \item Un email est envoyé au manager (si SMTP configuré)
    \item L'employé peut annuler sa demande
\end{itemize}

\subsection{Validation par le Manager}

\begin{figure}[H]
    \centering
    % \includegraphics[width=0.95\textwidth]{images/manager_validation.png}
    \caption{Interface du manager avec options de validation}
    \label{fig:manager}
\end{figure}

Le manager dispose de deux options principales :
\begin{itemize}
    \item \textbf{Valider} — Approuve la demande et l'envoie à la DAF
    \item \textbf{Refuser} — Ouvre un wizard pour saisir le motif de refus
\end{itemize}

\textbf{Note :} L'option \textbf{Remettre en brouillon} n'apparaît que pour les demandes déjà refusées, permettant à l'employé de corriger et resoumettre sa demande.

\subsection{Wizard de refus}

\begin{figure}[H]
    \centering
    % \includegraphics[width=0.7\textwidth]{images/wizard_refus.png}
    \caption{Popup de saisie du motif de refus}
    \label{fig:wizard}
\end{figure}

Le wizard de refus garantit qu'un motif est toujours fourni lors d'un refus :
\begin{itemize}
    \item Champ texte obligatoire pour le motif
    \item Bouton "Confirmer le refus" pour valider
    \item Bouton "Annuler" pour fermer sans action
    \item Le motif est ensuite visible dans la demande et envoyé par email
\end{itemize}

\subsection{État Approuvé}

\begin{figure}[H]
    \centering
    % \includegraphics[width=0.95\textwidth]{images/approuve.png}
    \caption{Demande approuvée par le manager}
    \label{fig:approuve}
\end{figure}

Après approbation :
\begin{itemize}
    \item L'état passe à \textbf{Approuvé}
    \item Tous les DAF reçoivent une activité de notification
    \item L'employé reçoit un email de confirmation
    \item Les DAF reçoivent un email détaillé pour traitement
    \item Les DAF peuvent prendre en charge la demande
\end{itemize}

\subsection{Prise en charge par la DAF}

\begin{figure}[H]
    \centering
    % \includegraphics[width=0.95\textwidth]{images/daf_encours.png}
    \caption{Interface DAF pour le traitement de la demande}
    \label{fig:daf}
\end{figure}

La DAF dispose de trois actions :
\begin{itemize}
    \item \textbf{Prendre en charge} — Passe l'état à "En cours" (achat de billets, réservations)
    \item \textbf{Refuser} — Permet de refuser la demande avec motif
    \item \textbf{Terminer} — Clôture définitive de la demande
\end{itemize}

\subsection{État Terminé}

\begin{figure}[H]
    \centering
    % \includegraphics[width=0.95\textwidth]{images/termine.png}
    \caption{Demande terminée et archivée}
    \label{fig:termine}
\end{figure}

L'état \textbf{Terminé} représente la clôture complète du processus :
\begin{itemize}
    \item Formulaire en lecture seule
    \item Aucun bouton d'action disponible
    \item Historique complet conservé
    \item Demande archivée pour consultation future
\end{itemize}

\subsection{Suivi du workflow (Chatter)}

\begin{figure}[H]
    \centering
    % \includegraphics[width=0.95\textwidth]{images/chatter.png}
    \caption{Historique complet des actions dans le Chatter (bas du formulaire)}
    \label{fig:chatter}
\end{figure}

Le \textbf{Chatter}, situé en bas de chaque formulaire de demande, enregistre automatiquement l'historique complet de toutes les actions effectuées. Comme illustré dans la figure \ref{fig:chatter}, chaque événement du workflow est tracé avec précision :

\begin{itemize}
    \item \textbf{Date et heure} — Horodatage précis de chaque action (Nov 15, 2:55 AM / 3:00 AM)
    \item \textbf{Subject} — Description de l'événement :
    \begin{itemize}
        \item "Demande DEP00001 - À valider" — Message d'activité pour le manager
        \item "[Déplacement] Demande \${object.name\} soumise" — Email de notification envoyé
        \item "[Déplacement] Demande \${object.name\} approuvée" — Confirmation d'approbation
        \item "[Déplacement] Demande \${object.name\} à traiter" — Notification à la DAF
    \end{itemize}
    \item \textbf{Author} — Utilisateur ayant déclenché l'action (Sophie Dubois, Marie Martin)
    \item \textbf{Related Document} — Lien vers la demande concernée (DEP00001)
\end{itemize}

Cette section du Chatter combine plusieurs types de messages :
\begin{itemize}
    \item \textbf{Messages système} — Créés automatiquement lors des transitions d'état
    \item \textbf{Emails envoyés} — Notifications aux managers et DAF
    \item \textbf{Activités} — Tâches assignées aux validateurs
    \item \textbf{Notes} — Commentaires manuels des utilisateurs (si ajoutés)
\end{itemize}

Cette traçabilité automatique offre :
\begin{itemize}
    \item \textbf{Audit complet} — Historique immuable chronologique de toutes les actions
    \item \textbf{Conformité} — Preuve de validation à chaque étape du processus
    \item \textbf{Transparence} — Visibilité sur qui a fait quoi et quand
    \item \textbf{Responsabilité} — Attribution claire des décisions et validations
\end{itemize}

Tous les followers de la demande (manager, DAF, employé) peuvent consulter cet historique pour suivre l'évolution de la demande en temps réel.

\subsection{Vues additionnelles}

Le module propose 7 types de vues différentes :

\subsubsection{Vue Liste}
\begin{itemize}
    \item Affichage tabulaire de toutes les demandes
    \item Tri et filtres dynamiques
    \item Totaux automatiques des montants
    \item Codes couleur selon l'état
\end{itemize}

\subsubsection{Vue Kanban}
\begin{itemize}
    \item Affichage en cartes groupées par état
    \item Glisser-déposer pour changer d'état (si droits suffisants)
    \item Informations synthétiques sur chaque carte
\end{itemize}

\subsubsection{Vue Calendrier}
\begin{itemize}
    \item Visualisation des déplacements dans le temps
    \item Groupement par employé (code couleur)
    \item Navigation mensuelle/hebdomadaire
\end{itemize}

\subsubsection{Vue Pivot}
\begin{itemize}
    \item Tableau croisé dynamique
    \item Analyse multidimensionnelle (employé × état × montant)
    \item Export Excel/CSV
\end{itemize}

\subsubsection{Vue Graph}
\begin{itemize}
    \item Graphiques en barres, courbes, secteurs
    \item Statistiques par état, employé, période
    \item Visualisation des tendances
\end{itemize}

\section{Données de démonstration}

\subsection{Villes préchargées}

Le module inclut des données de démonstration pour le Maroc :

\begin{table}[H]
\centering
\renewcommand{\arraystretch}{1.3}
\begin{tabular}{|l|l|l|}
\hline
\textbf{Ville} & \textbf{Pays} & \textbf{Code Postal} \\ \hline
Casablanca & Maroc & 20000 \\ \hline
Rabat & Maroc & 10000 \\ \hline
Marrakech & Maroc & 40000 \\ \hline
Fès & Maroc & 30000 \\ \hline
Tanger & Maroc & 90000 \\ \hline
Agadir & Maroc & 80000 \\ \hline
\end{tabular}
\caption{Villes préchargées}
\label{tab:villes}
\end{table}

\subsection{Véhicules de service}

\begin{table}[H]
\centering
\renewcommand{\arraystretch}{1.3}
\begin{tabular}{|l|l|l|l|}
\hline
\textbf{Nom} & \textbf{Marque} & \textbf{Modèle} & \textbf{Immatriculation} \\ \hline
Peugeot 208 & Peugeot & 208 & ABC-123-MA \\ \hline
Renault Clio & Renault & Clio & XYZ-456-MA \\ \hline
Dacia Logan & Dacia & Logan & DEF-789-MA \\ \hline
Toyota Corolla & Toyota & Corolla & GHI-012-MA \\ \hline
Volkswagen Golf & Volkswagen & Golf & JKL-345-MA \\ \hline
\end{tabular}
\caption{Véhicules de service préchargés}
\label{tab:vehicules}
\end{table}

\section{Déploiement avec Docker}

\subsection{Configuration Docker Compose}

Le projet utilise Docker pour faciliter le déploiement et garantir la portabilité :

\begin{lstlisting}[language=bash, caption=docker-compose.yml]
version: "3.8"

services:
  odoo:
    image: odoo:19.0
    container_name: odoo-web
    ports:
      - "8069:8069"
    environment:
      - HOST=host.docker.internal
      - USER=odoo
      - PASSWORD=odoo
    volumes:
      - odoo-web-data:/var/lib/odoo
      - ./addons:/mnt/extra-addons

volumes:
  odoo-web-data:
\end{lstlisting}

\subsection{Installation et démarrage}

\begin{verbatim}
# 1. Démarrer les conteneurs
docker-compose up -d

# 2. Accéder à Odoo
http://localhost:8069

# 3. Créer la base de données
- Database: odoo19
- Email: admin@example.com
- Password: admin

# 4. Installer le module
Apps → Rechercher "Gestion des Déplacements" → Installer

# 5. Redémarrer si nécessaire
docker restart odoo-web
\end{verbatim}

\section{Tests et Validation}

\subsection{Scénarios de test}

\subsubsection{Test 1 : Création et soumission (Employé)}
\begin{enumerate}
    \item Se connecter en tant qu'employé
    \item Créer une nouvelle demande
    \item Vérifier le calcul automatique du manager
    \item Remplir tous les champs obligatoires
    \item Vérifier le calcul du nombre de jours
    \item Vérifier le calcul du montant (national vs international)
    \item Joindre un ordre de mission PDF
    \item Soumettre la demande
    \item Vérifier que l'état passe à "Soumis"
    \item Vérifier la création de l'activité pour le manager
\end{enumerate}

\subsubsection{Test 2 : Validation (Manager)}
\begin{enumerate}
    \item Se connecter en tant que manager
    \item Vérifier la notification d'activité 
    \item Ouvrir la demande soumise
    \item Tester l'approbation :
    \begin{itemize}
        \item Cliquer sur "Valider"
        \item Vérifier le passage à l'état "Approuvé"
        \item Vérifier la création d'activités pour les DAF
        \item Vérifier l'envoi des emails
    \end{itemize}
    \item Tester le refus (sur une autre demande) :
    \begin{itemize}
        \item Cliquer sur "Refuser"
        \item Vérifier l'ouverture du wizard
        \item Saisir un motif obligatoire
        \item Confirmer le refus
        \item Vérifier le passage à l'état "Refusé"
        \item Vérifier l'affichage du motif
    \end{itemize}
\end{enumerate}

\subsubsection{Test 3 : Traitement (DAF)}
\begin{enumerate}
    \item Se connecter en tant que DAF
    \item Vérifier la notification d'activité
    \item Ouvrir une demande approuvée
    \item Cliquer sur "Prendre en charge"
    \item Vérifier le passage à "En cours"
    \item Cliquer sur "Terminer"
    \item Vérifier le passage à "Terminé"
    \item Vérifier que tous les boutons disparaissent
\end{enumerate}

\subsubsection{Test 4 : Contraintes de validation}
\begin{enumerate}
    \item Tester date\_fin < date\_debut → Erreur
    \item Tester distance = 0 → Erreur
    \item Tester avion avec distance < 500 km → Erreur
    \item Tester mode = véhicule\_service sans véhicule → Erreur
    \item Tester soumission sans ordre de mission → Erreur
\end{enumerate}

\subsection{Résultats des tests}

\begin{table}[H]
\centering
\renewcommand{\arraystretch}{1.3}
\begin{tabular}{|p{7cm}|p{3cm}|p{4cm}|}
\hline
\textbf{Test} & \textbf{Résultat} & \textbf{Remarques} \\ \hline
Calcul automatique du manager & OK & Basé sur parent\_id \\ \hline
Calcul du nombre de jours & OK & Inclusif (fin - début + 1) \\ \hline
Détection international & OK & Compare country\_id \\ \hline
Calcul des frais & OK & 700 DH (nat) / 1500 DH (int) \\ \hline
Calcul classe avion & OK & Business si > 6000 km \\ \hline
Notifications activités & OK & Affichage dans horloge \\ \hline
Messages Chatter & OK & Historique complet \\ \hline
Emails & Optionnel & Nécessite SMTP \\ \hline
Wizard refus & OK & Motif obligatoire \\ \hline
Contraintes validation & OK & Toutes les règles OK \\ \hline
Droits d'accès & OK & Record rules OK \\ \hline
\end{tabular}
\caption{Résultats des tests fonctionnels}
\label{tab:tests}
\end{table}

\section{Points forts et limitations}

\subsection{Points forts}

\begin{itemize}
    \item \textbf{Architecture modulaire} — Code bien organisé et maintenable
    \item \textbf{Workflow robuste} — Transitions d'état validées et sécurisées
    \item \textbf{Calculs automatiques} — Réduction des erreurs de saisie
    \item \textbf{Notifications multicouches} — Activités + Chatter + Emails
    \item \textbf{Sécurité granulaire} — Record rules et groupes bien définis
    \item \textbf{Traçabilité complète} — Historique immuable des actions
    \item \textbf{Gestion d'erreurs} — Emails non bloquants avec try-except
    \item \textbf{Interface intuitive} — 7 types de vues différentes
    \item \textbf{Données de démo} — Facilite les tests et la formation
    \item \textbf{Dockerisation} — Déploiement simplifié et portable
\end{itemize}

\subsection{Limitations et améliorations possibles}

\begin{itemize}
    \item \textbf{Calcul des frais simplifié} — Barèmes fixes non paramétrables
    \item \textbf{Pas de gestion de disponibilité} — Véhicules sans calendrier
    \item \textbf{Emails optionnels} — Nécessite configuration SMTP externe
    \item \textbf{Classe voyage non affichée dans emails} — Champ calculé non utilisé
    \item \textbf{Search view commentée} — Filtres et groupements désactivés
    \item \textbf{Pas de validation budgétaire} — Aucune limite de dépense
    \item \textbf{Pas de rapports PDF} — Génération d'ordres de mission manuelle
\end{itemize}

\subsection{Évolutions futures envisageables}

\begin{itemize}
    \item \textbf{Table de barèmes configurables} — Par pays, catégorie employé, etc.
    \item \textbf{Calendrier de disponibilité des véhicules} — Éviter les conflits
    \item \textbf{Intégration comptable} — Liaison avec module Accounting
    \item \textbf{Génération automatique de PDF} — Ordres de mission et rapports
    \item \textbf{Validation budgétaire} — Contrôle des enveloppes par département
    \item \textbf{Dashboard analytique} — KPI et statistiques en temps réel
    \item \textbf{API REST} — Intégration avec systèmes externes
    \item \textbf{Application mobile} — Soumission de demandes depuis smartphone
\end{itemize}

\section{Conclusion}

\subsection{Bilan du projet}

Ce projet a permis de développer un module Odoo 19 complet et fonctionnel pour la gestion des déplacements professionnels. L'objectif principal — automatiser le workflow de validation à trois niveaux (Employé → Manager → DAF) — a été atteint avec succès.

Le module répond aux exigences suivantes :
\begin{itemize}
    \item  Création et soumission de demandes par les employés
    \item  Validation hiérarchique avec motif de refus obligatoire
    \item  Calculs automatiques (jours, frais, classe voyage)
    \item  Notifications multicouches (activités, chatter, emails)
    \item  Sécurité granulaire avec record rules
    \item  Traçabilité complète via Chatter
    \item  Interface utilisateur intuitive avec 7 vues
\end{itemize}

\subsection{Compétences acquises}

Ce projet a permis d'approfondir les compétences suivantes :
\begin{itemize}
    \item Développement Odoo (modèles, vues, workflows, sécurité)
    \item Python pour la logique métier et les calculs automatiques
    \item XML pour les vues et les données
    \item Système de notifications Odoo (activités, chatter, followers)
    \item Gestion de la sécurité et des droits d'accès
    \item Dockerisation d'applications ERP
    \item Conception de workflows d'approbation
\end{itemize}

\subsection{Perspectives}

Le module constitue une base solide pour la gestion des déplacements et peut être enrichi progressivement selon les besoins de l'entreprise. Les évolutions futures pourront porter sur l'intégration comptable, la génération automatique de documents PDF, ou encore le développement d'une application mobile pour les employés en déplacement.

Ce projet démontre la puissance et la flexibilité d'Odoo pour développer rapidement des solutions métier sur mesure, tout en bénéficiant de l'écosystème riche du framework (mail, HR, notifications, sécurité).

\end{document}
